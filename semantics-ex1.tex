\documentclass[10pt,a4paper]{exam}
\usepackage[latin1]{inputenc}
\usepackage{amsmath}
\usepackage{amsfonts}
\usepackage{amssymb}
\usepackage{graphicx}
\usepackage{titlesec}
\usepackage{hyperref}
\usepackage{fancyeq}
\usepackage{tikz}
%\usepackage{tikz-uml}
\usepackage{mathpartir}
\usetikzlibrary{shapes,arrows,backgrounds,positioning}
\usepackage{graphicx,xcolor}
\usepackage{geometry}
\usepackage{everysel}


\tikzset{
  treenode/.style = {align=center, inner sep=0pt, text centered,
    font=\sffamily},
  arn_n/.style = {treenode, circle, black, font=\sffamily\bfseries, draw=black,
    fill=white, text width=1.5em},% arbre rouge noir, noeud noir
  arn_r/.style = {treenode, circle, red, draw=red, 
    text width=1.5em, very thick},% arbre rouge noir, noeud rouge
  arn_x/.style = {treenode, rectangle, draw=black,
    minimum width=0.5em, minimum height=0.5em}% arbre rouge noir, nil
}

\usepackage[sc]{mathpazo}
\linespread{1.05}         % Palatino needs more leading (space between lines)
\usepackage[T1]{fontenc}

% some format settings
% for hard-bound final submission, use:
%\setlength{\oddsidemargin}{4.6mm}     % 30 mm left margin - 1 in
% for soft-bound version and techreport, use instead:

\setlength{\oddsidemargin}{-0.4mm}    % 25 mm left margin - 1 in
\setlength{\evensidemargin}{\oddsidemargin}
\setlength{\topmargin}{-5.4mm}        % 20 mm top margin - 1 in
\setlength{\textwidth}{160mm}         % 20/25 mm right margin
\setlength{\textheight}{237mm}        % 20 mm bottom margin
\setlength{\headheight}{5mm}
\setlength{\headsep}{5mm}
\setlength{\parindent}{0mm}
\setlength{\parskip}{\medskipamount}
\renewcommand\baselinestretch{1.2} % thesis format (not needed for techreport)
% don't let large figures hijack entire pages
\renewcommand\topfraction{.9}
\renewcommand\textfraction{.1}
\renewcommand\floatpagefraction{.8}

\pagestyle{headandfoot}
%\pointsinrightmargin
%\pointname{ marks}
%\marginpointname{ marks}

\marksnotpoints 

\definecolor{campurple}{HTML}{862D91} 
\definecolor{campurpledark}{HTML}{2A185C}

\hypersetup{  
  urlcolor=campurple,
  linkcolor=campurple,
  colorlinks=true  
}

\titlelabel{\llap{\thetitle\quad}}

\newcommand {\lbrac} {\makebox[0pt]{[\kern-1ex[}}
\newcommand {\rbrac} {\makebox[0pt]{]\kern-1ex]}}
\newcommand{\denote}[1]{\lbrac~#1~\rbrac}


\def\mystrut(#1,#2){\vrule height #1pt depth #2pt width 0pt} 

\titlespacing*{\section}{0pt}{0pt}{0pt}


\begin{document}

\newcommand{\course}{Semantics of Programming Languages}
\newcommand{\week}{1}
\newcommand{\topics}{Transition Systems and Types}

\everymath{\color{campurpledark}}
\everydisplay{\color{campurpledark}}

\vspace{-15pt}

%\begin{center}
%\emph{Complete SECTION 1 and ONE other section.}
%\end{center}

%\begin{center}
%\emph{Answer SECTION 1 and TWO other sections.}
%\end{center}

\marksnotpoints
\pointsdroppedatright
\marksnotpoints
\marginpointname{ \points}

\begin{center}
\Large {\color{campurpledark} \course} \\[-0.2cm]
\LARGE \textbf{\color{campurpledark} Exercise \week: \topics} \\
\end{center}

{\color{campurple}\hrule}

\vspace{0.5cm}

\begin{questions}

\section*{Transition Systems}

% hutton's razor with extensions
\question Consider the following grammar for the instruction set of a simple, stack-based abstract machine where $v \in \mathbb{N}$:
\begin{displaymath}
\begin{array}{lcl}
i & = & \mathbf{PUSH}~v \mid \mathbf{ADD}
\end{array}
\end{displaymath}
Additionally, we specify a grammar for programs. These are simply sequences of instructions where $\cdot$ denotes the empty program and $i;p$ denotes an instruction $i$ followed by a program $p$:
\begin{displaymath}
\begin{array}{lcl}
p & = & \cdot \mid i; p
\end{array}
\end{displaymath}
The operational semantics of the instructions $i$ have been specified informally as:
\begin{itemize}
\item $\mathbf{PUSH}~v$ pushes the value $v$ on top of the stack
\item $\mathbf{ADD}$ takes two values off the stack, adds them up, and then pushes the result on top of the stack
\end{itemize}
These descriptions are informal and could be interpreted incorrectly. Conveniently we have just learned how to better specify operational semantics! 
\begin{parts}
\part[1] Let a \emph{stack} be an element $s$ of $\mathbb{N}^*$ where $\varepsilon$ is the empty stack and sequences of natural numbers represent non-empty stacks. Suggest how \emph{configurations} should be represented in a transition system for the abstract machine described above. \droppoints 
\part[3] What is included in the set of values for this transition system? Discuss how this compares with the set of values for L1 from the notes. \droppoints 
\part[4] Give a formal, operational semantics for the abstract machine. \droppoints 
\part[2] Do there exist configurations which are stuck? If so, give an example. If not, prove it. \droppoints 
\end{parts}
\question Suppose that we add two new instructions
\begin{displaymath}
\begin{array}{lcl}
i & = & \ldots \mid \mathbf{LOAD}~\ell \mid \mathbf{STORE}~\ell
\end{array}
\end{displaymath}
where $\mathbf{LOAD}~\ell$ loads a value from a memory location labelled $\ell$ before pushing it on top of the stack and $\mathbf{STORE}~\ell$ pops a value off the stack and stores it in the memory location labelled $\ell$ afterwards. Labels are elements of a set $\mathbb{L}$ and memory locations store elements of $\mathbb{N}$.
\begin{parts}
\part[2] Choose an appropriate representation for \emph{stores} and show how the representation of \emph{configurations} needs to be changed to accommodate for them. \droppoints 
\part[4] Give a new operational semantics for the abstract machine. \droppoints 
\part[6] Show a complete derivation from an empty stack and empty store for the program 
\begin{displaymath}
\mathbf{PUSH}~4; \mathbf{PUSH}~8; \mathbf{STORE}~l; \mathbf{PUSH}~15; \mathbf{ADD}; \mathbf{LOAD}~l; \mathbf{ADD}; \cdot
\end{displaymath} \droppoints 
\end{parts}

\section*{Types}

\question A type may be seen as an approximation of the value of the term it is assigned to. For example, if an arbitrary expression $e$ has type $\mathbf{int}$, then we know that after evaluating $e$ we will end up with a \emph{value} of type $\mathbf{int}$. We also know that if \emph{e.g.} a function $f$ expects an argument of type $\mathbf{bool}$, then $e$ is not a suitable argument for it. Types allow us to catch many different classes of errors at compile-time. 
%However, manually annotating expressions with types is cumbersome, so we would like a program to do it for us. Suppose that a programming language whose syntax contains type annotations is stripped of them, then \emph{type inference} (or \emph{type reconstruction}) describes the process of assigning suitable types to the untyped terms of the language to reconstruct the original, annotated terms. 

For this exercise, consider the following specification for a simple expression language. $n \in \mathbb{N}$ and $b \in \mathbb{B}$:
\begin{displaymath}
e = e + e \mid n \mid b \mid \mathbf{if}~e~\mathbf{then}~e~\mathbf{else}~e
\end{displaymath}
We define the semantics of this language \emph{denotationally}\footnote{There is an entire Part II course dedicated to this, but we will keep it simple.}. That is, instead of using inductively-defined transitions between states of an abstract machine, we define a total, recursive function which maps expressions to some set of values.

Before we define a denotational semantics for the above expression language, let us briefly define some prerequisite concepts. Given two arbitrary sets, $A$ and $B$, the \emph{tagged union} of the two is defined as:
\begin{displaymath}
A + B = \{~\mathbf{inl}~a \mid a \in A~\} \cup \{~\mathbf{inr}~b \mid b \in B~\}
\end{displaymath}  
In other words, a tagged union is simply a union where the elements are tagged with either $\mathbf{inl}$ or $\mathbf{inr}$ to indicate which set they originate from.

A subscript $\bot$ indicates that a set is ``lifted''. This means that we include a value $\bot$ (read as \emph{bottom} or \emph{undefined}) in the set. For example:
\begin{displaymath}
\begin{array}{lcl}
\mathbb{B}_{\bot} & = & \{~\bot, \mathit{true}, \mathit{false}~\}
\end{array}
\end{displaymath}
Now we can define the denotational semantics for our expression language as:
\begin{displaymath}
\begin{array}{lcl}
\denote{\cdot} & : & e \to (\mathbb{N} + \mathbb{B})_{\bot}\\
\denote{n}         & = & \mathbf{inl}~n \\
\denote{b}         & = & \mathbf{inr}~b \\
\denote{e_0 + e_1} & = & \left\{ \begin{array}{ll}
\mathbf{inl}~(x + y) & \text{if } ~\denote{e_0}~ = \mathbf{inl}~x \text{ and } ~\denote{e_1}~ = \mathbf{inl}~y \\
\bot & \text{otherwise}
\end{array}  \right.\\
\denote{\mathbf{if}~e_0~\mathbf{then}~e_1~\mathbf{else}~e_2} & = & \left\{\begin{array}{ll}
\denote{e_1} & \qquad \quad \; \text{if } ~\denote{e_0}~ = \mathbf{inr}~\mathit{true}\\
\denote{e_2} & \qquad \quad \; \text{if } ~\denote{e_0}~ = \mathbf{inr}~\mathit{false}\\
\bot         & \qquad \quad \; \text{otherwise}
\end{array}\right.
\end{array}
\end{displaymath}
Note that the codomain (\emph{i.e.} the set of values) of the evaluation function $~\denote{\cdot}~$ is $(\mathbb{N} + \mathbb{B})_{\bot}$. Therefore, we have:
\begin{displaymath}
(\mathbb{N} + \mathbb{B})_{\bot} = \set{\bot, \mathbf{inl}~0,\mathbf{inl}~1, \ldots, \mathbf{inr}~\mathit{true}, \mathbf{inr}~\mathit{false}}
\end{displaymath}
To summarise, the co-domain of the evaluation function is the set consisting of $\bot$, all natural numbers tagged with $\mathbf{inl}$, and all boolean values tagged with $\mathbf{inr}$. We use $\bot$ in the definition of the evaluation function to indicate failure: \emph{i.e.} things we cannot evaluate. In other words, $\bot$ means that the evaluation got stuck.

According to the grammar for $e$, an example of a valid expression would be the following:
\begin{displaymath}
\mathbf{if}~4~\mathbf{then}~\mathit{false}~\mathbf{else}~8
\end{displaymath}
This expression does intuitively not make much sense. If we apply the evaluation function, it will evaluate to $\bot$. It gets stuck. In practice, it is not feasible to run a program to find out if it will get stuck or not. To catch such errors at compile-time, we will add a type system to our language. For this, we need to define suitable types. A good start would be one type for each kind of value in our language: 
\begin{displaymath}
\tau = \mathbf{nat} \mid \mathbf{bool}
\end{displaymath}
We know that a well-formed expression will either evaluate to a natural number or to a boolean value. Since types are approximations of the values of expressions, our definition of $\tau$ seems adequate.

%We now wish to define a binary relation $\vdash e : \tau$ which relates expressions $e$ to types $\tau$. For example, we want to say that all numbers in our language are of type $\mathbf{nat}$: $\vdash n : \mathbf{nat}$. Enumerating all of these relations would be impossible, however, so we declare rules according to which types are assigned different forms of expressions instead. We call this the \emph{deductive system}. Each rule in this system is presented in the following way:


%The type inference problem can now be formalised as: given an arbitrary expression $e$, can we assign a type $\tau$ to it such that the typing relation holds:
%\begin{displaymath}
%\vdash e : \tau
%\end{displaymath}
\newcommand{\metavar}[1]{{\color{campurple}#1}}
\begin{parts}
\part[5] Define a binary typing relation $ \vdash e : \tau$ for our simple expression language using a system of type inference rules of form:
\begin{mathpar}
    \inferrule*[right=\color{black}T-NAME]
    { assumption_0 \\ \ldots \\ assumption_n }
    { goal }\\
\end{mathpar}
\emph{Hint}: both of the branches of a conditional expression should have the same type. \droppoints
\part[3] Prove using your inference rules that $\mathbf{if}~\mathit{true}~\mathbf{then}~(4+8)~\mathbf{else}~23$ is well-typed. \droppoints 
\part[2] Explain why $\mathbf{if}~4~\mathbf{then}~\mathit{false}~\mathbf{else}~8$
cannot be well-typed. \droppoints 
\part There are two problems which we typically need to solve for the type system of a programming language:
\begin{itemize}
    \item Given some expression $e$ and a type $\tau$, \emph{type checking} asks: is $\tau$ a suitable type for $e$?
    \item Given only some expression $e$, \emph{type inference} asks: what is a suitable type $\tau$ for $e$?
\end{itemize} 
Answer the following questions about type checking and type inference: 
\begin{subparts}
\subpart[4] Discuss to what extent your typing judgements address these problems. \droppoints
\subpart[4] Suggest how the problem of type checking could be reduced to type inference. What assumptions do you need to make about your types? \droppoints  
\end{subparts} 
\part[6] Turn your typing judgements into an algorithm for type inference and implement it in a language of your choice, but not ML\footnote{Last year I received some particularly elegant solutions in Haskell and Prolog.}. \droppoints 
\part[6] Translate the denotational semantics shown above into an operational semantics. Discuss how your answer compares with the denotational semantics. \droppoints
\part[10] Suppose that we want to allow each branch of a conditional expression to return a different type. Discuss how this could be implemented in the type system. Justify your decisions and describe all changes you would make to the types and the typing rules. \droppoints
\part[10] Can you come up with a type system for which type inference is undecidable, but type checking is decidable? This question is particularly difficult and you should not spend too much time on it. \droppoints 
\end{parts}
\end{questions}
\end{document}