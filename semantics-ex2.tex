\documentclass[10pt,a4paper]{exam}
\usepackage[latin1]{inputenc}
\usepackage{amsmath}
\usepackage{amsfonts}
\usepackage{amssymb}
\usepackage{graphicx}
\usepackage{titlesec}
\usepackage{hyperref}
\usepackage{fancyeq}
\usepackage{tikz}
%\usepackage{tikz-uml}
\usepackage{mathpartir}
\usetikzlibrary{matrix,decorations.pathmorphing,shapes,arrows,backgrounds,positioning}
\usepackage{graphicx,xcolor}
\usepackage{geometry}
\usepackage{everysel}
\usepackage{amsthm}

\tikzset{
  treenode/.style = {align=center, inner sep=0pt, text centered,
    font=\sffamily},
  arn_n/.style = {treenode, circle, black, font=\sffamily\bfseries, draw=black,
    fill=white, text width=1.5em},% arbre rouge noir, noeud noir
  arn_r/.style = {treenode, circle, red, draw=red, 
    text width=1.5em, very thick},% arbre rouge noir, noeud rouge
  arn_x/.style = {treenode, rectangle, draw=black,
    minimum width=0.5em, minimum height=0.5em}% arbre rouge noir, nil
}

\usepackage[sc]{mathpazo}
\linespread{1.05}         % Palatino needs more leading (space between lines)
\usepackage[T1]{fontenc}

% some format settings
% for hard-bound final submission, use:
%\setlength{\oddsidemargin}{4.6mm}     % 30 mm left margin - 1 in
% for soft-bound version and techreport, use instead:

\setlength{\oddsidemargin}{-0.4mm}    % 25 mm left margin - 1 in
\setlength{\evensidemargin}{\oddsidemargin}
\setlength{\topmargin}{-5.4mm}        % 20 mm top margin - 1 in
\setlength{\textwidth}{160mm}         % 20/25 mm right margin
\setlength{\textheight}{237mm}        % 20 mm bottom margin
\setlength{\headheight}{5mm}
\setlength{\headsep}{5mm}
\setlength{\parindent}{0mm}
\setlength{\parskip}{\medskipamount}
\renewcommand\baselinestretch{1.2} % thesis format (not needed for techreport)
% don't let large figures hijack entire pages
\renewcommand\topfraction{.9}
\renewcommand\textfraction{.1}
\renewcommand\floatpagefraction{.8}

\pagestyle{headandfoot}
%\pointsinrightmargin
%\pointname{ marks}
%\marginpointname{ marks}

\marksnotpoints 

\definecolor{campurple}{HTML}{862D91} 
\definecolor{campurpledark}{HTML}{2A185C}

\hypersetup{  
  urlcolor=campurple,
  linkcolor=campurple,
  colorlinks=true  
}

\titlelabel{\llap{\thetitle\quad}}

\newcommand {\lbrac} {\makebox[0pt]{[\kern-1ex[}}
\newcommand {\rbrac} {\makebox[0pt]{]\kern-1ex]}}
\newcommand{\denote}[1]{\lbrac~#1~\rbrac}


\def\mystrut(#1,#2){\vrule height #1pt depth #2pt width 0pt} 

\titlespacing*{\section}{0pt}{0pt}{0pt}


\begin{document}

\newcommand{\course}{Semantics of Programming Languages}
\newcommand{\week}{2}
\newcommand{\topics}{Induction and Functions}

\everymath{\color{campurpledark}}
\everydisplay{\color{campurpledark}}

%\vspace{-15pt}

%\begin{center}
%\emph{Complete SECTION 1 and ONE other section.}
%\end{center}

%\begin{center}
%\emph{Answer SECTION 1 and TWO other sections.}
%\end{center}

\marksnotpoints
\pointsdroppedatright
\marksnotpoints
\marginpointname{ \points}

\begin{center}
\Large {\color{campurpledark} \course} \\[-0.2cm]
\LARGE \textbf{\color{campurpledark} Exercise \week: \topics} \\
\end{center}

{\color{campurple}\hrule}

\newcommand{\metavar}[1]{{\color{campurple}#1}}

\theoremstyle{definition}
\newtheorem{mydef}{Definition}

\begin{questions}

\section*{Induction}

\question For the previous supervision, we developed formal descriptions of the operational semantics and type inference rules of some simple programming languages. Such formal specifications allow us to prove properties about the behaviour of programs which are written in those languages. 
\begin{parts}
%\part[10] Below is a model answer for the operational semantics that was required in Question 2 of the last set of exercises:  
%\begin{mathpar}
%\inferrule*[right=E-ADD]{ }
%{\langle \mathbf{ADD}; p, xys, \sigma \rangle \longrightarrow \langle p, (x+y)s, \sigma \rangle} \and
%\inferrule*[right=E-PUSH]{ }
%{\langle \mathbf{PUSH}~v; p, s, \sigma \rangle \longrightarrow \langle p, vs, \sigma \rangle} \\
%\inferrule*[right=E-LOAD]{ \ell \in \mathit{dom}(\sigma) }
%{\langle \mathbf{LOAD}~\ell; p, s, \sigma \rangle \longrightarrow \langle p, vs, \sigma \rangle} \and
%\inferrule*[right=E-STORE]{ }
%{\langle \mathbf{STORE}~\ell; p, vs, \sigma \rangle \longrightarrow \langle p, s, \sigma + \set{ \ell \mapsto v } \rangle}
%\end{mathpar}
%Below is a definition of the determinism property for programs in this language:
% \droppoints 
\part Below is a model answer for the operational semantics and typing judgements which were required in Question 3 of the last set of exercises:
\begin{mathpar}
\inferrule*[right=\color{black}E-IF]
{\metavar{e_0} \longrightarrow \metavar{e'_0}}
{\mathbf{if}~\metavar{e_0}~\mathbf{then}~{\color{campurple}e_1}~\mathbf{else}~{\color{campurple}e_2} \longrightarrow \mathbf{if}~\metavar{e'_0}~\mathbf{then}~{\color{campurple}e_1}~\mathbf{else}~{\color{campurple}e_2}}\\
\inferrule*[right=\color{black}E-IF-TRUE]
{ }
{\mathbf{if}~\mathit{true}~\mathbf{then}~{\color{campurple}e_1}~\mathbf{else}~{\color{campurple}e_2} \longrightarrow {\color{campurple}e_1}}\and
\inferrule*[right=\color{black}E-IF-FALSE]
{ }
{\mathbf{if}~\mathit{false}~\mathbf{then}~{\color{campurple}e_1}~\mathbf{else}~{\color{campurple}e_2} \longrightarrow {\color{campurple}e_2}}\\
\inferrule*[right=\color{black}E-PLUS]
{ \metavar{n} = \metavar{n_0} + \metavar{n_1} }
{\metavar{n_0} + \metavar{n_1} \longrightarrow \metavar{n}}\and
\inferrule*[right=\color{black}E-PLUS-LEFT]
{ \metavar{e_0} \longrightarrow \metavar{e'_0} }
{\metavar{e_0} + \metavar{e_1} \longrightarrow \metavar{e'_0} + \metavar{e_1}}\and
\inferrule*[right=\color{black}E-PLUS-RIGHT]
{ \metavar{e_1} \longrightarrow \metavar{e'_1} }
{\metavar{n_0} + \metavar{e_1} \longrightarrow \metavar{n_0} + \metavar{e'_1}}
\end{mathpar}
\begin{mathpar}
\inferrule*[right=T-NAT]{ }
{\vdash n : \mathbf{nat}} \and
\inferrule*[right=T-BOOL]{ }
{\vdash b : \mathbf{bool}} \\
\inferrule*[right=T-PLUS]{ \vdash e_0 : \mathbf{nat} \\ \vdash e_1 : \mathbf{nat} }
{\vdash e_0 + e_1 : \mathbf{nat}} \and
\inferrule*[right=T-COND]{ \vdash e_0 : \mathbf{bool} \\ \vdash e_1 : \tau \\ \vdash e_2 : \tau}
{\vdash \mathbf{if}~e_0~\mathbf{then}~e_1~\mathbf{else}~e_2 : \tau}
\end{mathpar}
\begin{subparts}
\subpart[8] The operational semantics for this language are deterministic if there is no more than one possible reduction for every configuration:
\begin{displaymath}
\begin{array}{lcl}
\Phi(e) & \overset{\mathit{def}}{=} & \forall e',e''.  e  \longrightarrow  e'  \wedge  e  \longrightarrow  e''  \Rightarrow  e'  =  e'' 
\end{array}  
\end{displaymath}
Prove, either by hand in as much detail as you deem necessary or using a proof assistant of your choice, that $\forall e.\Phi(e)$. \droppoints 
\subpart[8] The type system for this language has the progress property if every well-typed term is either a value or can be reduced further:
\begin{displaymath}
\begin{array}{lcl}
\Phi(e, \tau) & \overset{\mathit{def}}{=} & \vdash e : \tau \Rightarrow e \in \mathbb{V} \vee \exists e'. e  \longrightarrow  e' 
\end{array}
\end{displaymath}
where $\mathbb{V}$ is the set of values. Prove that this property holds. \droppoints 
\subpart[8] The type system for this language has the preservation property if the every well-typed term which is not a value maintains its type after further reduction:
\begin{displaymath}
\begin{array}{lclcl}
\Phi(e,\tau,e') & \overset{\mathit{def}}{=} & \vdash e : \tau \wedge e \longrightarrow e' & \Rightarrow & \vdash e' : \tau 
\end{array}
\end{displaymath}
Prove that this language has the type preservation property. \droppoints 
\subpart[8] This language is safe if reducing a well-typed term by an arbitrary number of steps either results in a value or a term which can be reduced further:
\begin{displaymath}
\begin{array}{lclcl}
\Phi(e,\tau,e') & \overset{\mathit{def}}{=} & \vdash e : \tau \wedge e \longrightarrow^* e' & \Rightarrow & e' \in \mathbb{V} \vee \exists e''.e' \longrightarrow e''
\end{array} 
\end{displaymath}
Prove that this property holds. \emph{Hint}: by induction on $\longrightarrow^*$, beginning with the case where evaluation results in a value in one step. \droppoints 
\subpart[5] People often say that ``well-typed programs do not go wrong''. Discuss what is meant by this with reference to the above properties. \droppoints 
\end{subparts}
\end{parts} 

\section*{Functions}

\question The $\color{black}\lambda$-calculus is a formal system in which computation is described using functions and substitution. Among other things, it serves as the theoretical foundation for programming languages. Functional programming languages in particular may be seen as just the $\color{black}\lambda$-calculus with a large amount of syntactic sugar. The syntax of the $\lambda$-calculus is shown below as a context-free grammar:
\begin{displaymath}
\begin{array}{lcl}
e & \to & x \mid \lambda x.e \mid e~e
\end{array}
\end{displaymath} 
An expression $e$ is either a variable $x$ from a set of variables $\mathbb{V}$, an abstraction\footnote{\emph{i.e.} a function} which binds a variable $x$ in a sub-expression $e$, or an application of one expression to another. Note that function application is left associative. Variables are either \emph{free} or \emph{bound} depending on whether there is an abstraction which binds them in the surrounding scope (\emph{i.e.} further on the left). Formally, the set of free variables of some expression is given by: 
\begin{displaymath}
\begin{array}{lcl}
\mathit{fvs}(x) & = & \set{x} \\
\mathit{fvs}(\lambda x.e) & = & \mathit{fvs}(e) \setminus \set{x} \\
\mathit{fvs}(e_0~e_1) & = & \mathit{fvs}(e_0) \cup \mathit{fvs}(e_1)
\end{array}
\end{displaymath}
\begin{parts}
\part[1] What are the free variables of $\lambda c.\lambda a.\lambda k.\lambda e.(i~(\lambda s.a~l))~(\lambda i.e)$? \droppoints 
\part Expressions in the $\color{black}\lambda$-calculus are considered equivalent up to renaming. This is known as $\color{black}\alpha$-equivalence. Formally:
\begin{displaymath}
\begin{array}{lcll}
\lambda x.e & \underset{\alpha}{\Leftrightarrow} & \lambda y.[x \mapsto y]e & \mathbf{where}~y \not \in \mathit{fvs}(e)
\end{array}
\end{displaymath} 
The semantics are given by a single reduction rule, known as $\color{black}\beta$-reduction:
\begin{displaymath}
\begin{array}{lcl}
(\lambda x.e_0)~e_1 & \underset{\beta}{\Rightarrow} & [x \mapsto e_1]e_0
\end{array}
\end{displaymath}
Substitution is \emph{capture-avoiding}: free variables of an expression which is to be substituted into another need to be renamed if they would otherwise become bound.
\begin{displaymath}
\begin{array}{lcl}
[x \mapsto s]y & = & \left\{ \begin{array}{ll}
s & \mathbf{if}~x \equiv y\\ 
y & \mathbf{otherwise}
\end{array}\right.
 \\[0cm] 
[x \mapsto s](e_0~e_1) & = & ([x \mapsto s]e_0)~([x \mapsto s]e_1) \\[0cm]
 [x \mapsto s](\lambda y.e) & = & \left\{ \begin{array}{ll}
\lambda y.e & \mathbf{if}~x \equiv y\\ 
\lambda y.[x \mapsto s]e & \mathbf{if}~x \not \equiv y \wedge y \not \in \mathit{fvs}(s) \\
\lambda z.[x \mapsto s]([y \mapsto z]e) & \mathbf{if}~x \not \equiv y \wedge y \in \mathit{fvs}(s), \mathbf{where}~z~\text{is fresh}
\end{array} \right. 
\end{array}
\end{displaymath}
Finally, $\color{black} \eta$-conversion allows us to abstract over part of an expression:
\begin{displaymath}
\begin{array}{lcll}
f & \underset{\eta}{\Leftrightarrow} & \lambda x.f~x & \mathbf{where}~x \not \in \mathit{fvs}(f)
\end{array}
\end{displaymath}
For each of the following pairs of expressions, give step-by-step instructions for how one may be transformed into the other using $\alpha$-equality, $\beta$-reduction, and $\eta$-conversion.
\begin{subparts}
\subpart[1] $\lambda x.x$ to $\lambda y.y$ \droppoints 
\subpart[1] $\lambda x.(\lambda f.f~x)~g$ to $\lambda y.g~y$ \droppoints 
\subpart[1] $f~x$ to $(\lambda y.f~y)~x$ \droppoints  
\subpart[1] $(\lambda f.\lambda x.f~(f~x))~(\lambda g.g)~z$ to $\lambda q.p~q$ \droppoints 
\end{subparts}
\part[4] Give call-by-value operational semantics for the $\lambda$-calculus. \droppoints 
\part[2] Give call-by-name operational semantics for the $\lambda$-calculus. \droppoints
\part Data can be encoded in the $\color{black} \lambda$-calculus using the Church encoding. For example, natural numbers are encoded using repeated function application, where $s$ is the successor function and $z$ represents 0:
\begin{displaymath}
\begin{array}{lcl}
\mathit{ZERO} & \overset{\mathit{def}}{=} & \lambda s.\lambda z.z \\
\mathit{ONE} & \overset{\mathit{def}}{=} & \lambda s.\lambda z.s~z \\
\mathit{TWO} & \overset{\mathit{def}}{=} & \lambda s.\lambda z.s~(s~z) \\
 & \ldots & 
\end{array}
\end{displaymath} 
\begin{subparts}
\subpart[2] Give a definition for $\mathit{PLUS}$ which, given two natural numbers in Church encoding, returns the sum. \emph{E.g.} $\mathit{PLUS}~\mathit{ONE}~\mathit{ONE} \underset{\beta}{\Rightarrow}^* \mathit{TWO}$. \droppoints 
\subpart[2] Give a definition for $\mathit{MUL}$ which, given two natural numbers in Church encoding, returns the product. \emph{E.g.} $\mathit{MUL}~\mathit{TWO}~\mathit{TWO} \underset{\beta}{\Rightarrow}^* \mathit{FOUR}$. \droppoints 
\end{subparts}
\part Recursion can be encoded in the $\color{black} \lambda$-calculus using a fixed-point combinator. The most common one is the Y-combinator:
\begin{displaymath}
\begin{array}{lcl}
Y & \overset{\mathit{def}}{=} & \lambda f.(\lambda x.f~(x~x))~(\lambda x.f~(x~x))
\end{array}
\end{displaymath}
Consider the following example. For an arbitrary expression $g$:
\begin{displaymath}
\begin{array}{cl}
\expr{Y~g} 
\hint{definition of $Y$}
\expr{(\lambda f.(\lambda x.f~(x~x))~(\lambda x.f~(x~x)))~g}
\hint{ $\beta$-reduction }
\expr{(\lambda x.g~(x~x))~(\lambda x.g~(x~x))}
\hint{ $\beta$-reduction }
\expr{g~((\lambda x.g~(x~x))~(\lambda x.g~(x~x)))}
\hint{ $\eta$-conversion }
\expr{g~((\lambda f.(\lambda x.f~(x~x))~(\lambda x.f~(x~x)))~g)}
\hint{definition of $Y$}
\lastexpr{g~(Y~g)} 
\end{array}
\end{displaymath}
Give a definition for $\mathit{INFTY}$ which represents $\infty$ using $Y$. \emph{Hint}: you may first want to define $\mathit{SUCC}$ which, given a single number in Church encoding, adds it to $\mathit{ONE}$. \droppoints 
\part[2] A formal system is \emph{strongly normalising} if and only if reducing a term will eventually lead to a term which cannot be reduced any further. In the $\lambda$-calculus, we refer to a part of an expression at which $\beta$-reduction may be applied as a $\beta$-redex.
\begin{displaymath}
\lambda y.(\lambda x.x)~y
\end{displaymath}
In the above expression, $(\lambda x.x)~y$ is a $\beta$-redex since it can be reduced to just $y$ using $\beta$-reduction. Using an example, discuss whether $\beta$-reduction will always eventually lead to expressions which do not contain any $\beta$-redexes. \droppoints 
\end{parts}

% simply-typed lambda calculus

\question The simply-typed $\lambda$-calculus (or $\lambda_{\to}$ for short) is a typed version of the $\lambda$-calculus which adds function types. Types in $\lambda_{\to}$ are defined over some set $B$ of base types as:
\begin{displaymath}
\begin{array}{lcl}
\tau & = & b \in B \mid \tau \to \tau 
\end{array}
\end{displaymath}
The syntax of $\lambda_{\to}$ is presented in Church-style with explicit type annotations on abstractions:
\begin{displaymath}
\begin{array}{lcl}
e & = & x \mid \lambda x : \tau.e \mid e~e
\end{array}
\end{displaymath}
Note that the syntax is otherwise identical to that of the pure $\lambda$-calculus. $\alpha$-equality, $\beta$-reduction, and $\eta$-expansion work just like they did before.
\begin{parts}
\part[6] Suggest appropriate typing judgements for $\lambda_{\to}$. \droppoints
\part[1] Modify your call-by-value operational semantics for the $\lambda$-calculus so that they can be used with $\lambda_{\to}$. \droppoints 
%\part[10] Prove that call-by-value $\lambda_{\to}$ is deterministic, has the progress property, and preserves types. \droppoints 
\part[5] Can the $Y$-combinator be typed in $\lambda_{\to}$? If so, show its type. If not, explain why. \droppoints 
\part[6] Annotating abstractions with types is tedious. Instead, we want to use the same terms as for the pure $\lambda$-calculus, \emph{i.e.} those without type annotations in abstractions, but we also want to be able to infer $\lambda_{\to}$ types for those terms. Discuss how you would implement a type inference algorithm which takes terms from the pure $\lambda$-calculus and returns $\lambda_{\to}$ types.  \droppoints 
\end{parts}
% simply typed lambda calculus?
% DNH?
% System F?
% Y combinator?

\end{questions}
\end{document}
